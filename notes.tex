\documentclass[10pt]{article}
\usepackage{a4wide}
\usepackage[english]{babel}
\usepackage{fancyhdr}
\usepackage{hyperref}
\usepackage{lastpage}
\usepackage{graphicx}
\usepackage[section]{placeins}
\usepackage[superscript,biblabel]{cite}
\usepackage[margin=1in]{geometry}

\pagestyle{fancy}
\rhead{}

\begin{document}
  \section*{Relational Data Model}
    \subsection*{Definitions}
    \begin{itemize}
      \item An entity class is a collection of entity instances that have a common structure.
      For example, a whole collection of student records could form an entity class.

      \item An entity instance represents a particular object of interest that is to be represented
      and tracked. For example, a student record is an entity instance that
      represents an individual student.

      \item An attribute represents a piece of interesting information, or a measurable fact,
      about the instances of an entitity class. For example, ’year of first registration’
      is a fact about students that might be represented as an attribute of all the
      instances of the entity class ’student’

      \item A domain is a set of values that can be assigned to an attribute; for example,
      the attribute ’birthday’ coudl be given values from the domain ’date’.

      \item A relationship is an association between entities. Entities are often identified by
      nouns in a requirements specification, and relationships by verbs. For example,
      ’owns’ might form a relationship between entities ’person’ and ’vehicle’. Relationships
      can be described as relationships between entity classes, or between
      entity instances.

      \item Mathematically, a relation consists of a heading, which is a subset of the Cartesian
      product of a set of (attribute name, domain) pairs, and a body, which contains
      (attribute name, value) pairs. For example, the entity class student could be
      represented as a heading, (student number, integer), (student name, text) and
      a body containing values like (student number, 123),(student name, Bloggs)
      A relation is implemented as a table in a relational database.

      \item A candidate key is a minimal set of attributes that identifies each individual
      row in a table (each tuple in a relation). For example, suppose there was a
      relation Slotroom,day,time in a timetabling application. Then room,day,time or
      class,day,time would serve as alternative candidate keys for the relation.

      \item The primary key is the candidate key that has been nominated to identify individual
      rows in a table. For example, in the timetabling relation above, room,day,time
      would be likely to form a suitable primary key because ’class’ is likely to change.
    \end{itemize}


  \section*{Database Integrity}
    \subsection*{ACID}
      \begin{itemize}
        \item Atomicity - something is either done completely, or not done at all. The state of doing it is not visible outside the database.
        \item Consistency - The database is in a legal state at all times. When a transaction occurs, it can not break the rules. These rules are about integrity, what is allowed and what is not allowed in certain locations of the database.
        \item Isolation - There can be more than one transaction occurring at the same time. A certain transaction will not see changes made by other transactions.
        \item Durability - When a transaction is done, it will be committed. After it is committed, it can no longer be undone.
      \end{itemize}

    \subsection*{Definitions}
      \begin{itemize}
        \item Enterprise rule:  empirical constraint on real world entities and attributes. Examples:  ``Each  pallet  contains  5184  bottles''  (a  real  example,  referring  to  supplies of an antiviral); a student is normally allowed at most two attempts at a module examination.
        \item Data  integrity:   the  data  in  a  database  models  the  real  world;  the  database corresponds to reality.  For example, a person has exactly one date of birth.  
        \item Integrity  constraint:  a  constraint  on  the  values  or  combinations  of  values  that are allowed to be entered into a database.  For example, `no two distinct vehicles are allowed to have the same vehicle identication number' is a constraint that allows `vehicle identication number' to identify a particular vehicle record.
        \item A domain is the set of possible values for an attribute; it is the type of the attribute. In a relational database, a domain is the set of possible values for the cells in a column of a table.
        \item A candidate key is a collection of attributes whose combined values are different for each tuple in a relation. A candidate key is also minimal in the sense that no subset of the candidate key will identify tuples in this way.
        \item A foreign key is a collection of attributes from one relation that constitutes a candidate key for another relation. The values of the foreign key attributes in the first relation must also be present in some tuple of the second relation.
      \end{itemize}

     \subsection*{Constraints}
       \begin{itemize}
         \item Other kinds of constraint are needed because domains and key constraints are not sufficient to capture all the different kinds of enterprise rule that need to be modelled in a database.
         \item For example, the constraint ‘every sheep farmer owns at least one sheep’ cannot be represented using domains, foreign keys and candidate keys.
         \item In general, minimal cardinality constraints (1..∗ cardinalities) demand more than domains and foreign and candidate keys.
       \end{itemize}

     \subsection*{SQL Integrity Checks}
       Enums: 
         \begin{verbatim}
         CREATE TYPE character_kind AS ENUM
           (’monster’, ’wizard’, ’hero’, ’seer’);
           CREATE TABLE character (
           ...,
           kind character_kind,
           etc
         );
         \end{verbatim}
       Check contraint:
         \begin{verbatim}
           CREATE TABLE character (
             ...,
             kind text CHECK
               (kind in (’monster’, ’wizard’, ’hero’, ’seer’)),
             etc
           );
         \end{verbatim}
        Foreign Key:
          \begin{verbatim}
            CREATE TABLE character_kind ( kind text primary key );
              INSERT INTO character_kind (kind) VALUES
                (monster), (wizard), (‘hero), (‘seer’);
              CREATE TABLE character (
              ...,
              kind text REFERENCES character_kind(kind)
            );
          \end{verbatim}
         Merits of these approaches:
         \begin{itemize}
            \item Enumerated type and foreign key reference allow constraint to be implemented once, and reused in many tables.
            \item Both these approaches allow a simple query to display values to be entered into the table via form widget.
            \item However, acceptable values are not immediately visible in the table definition when either of these approaches is used.
            \item Cannot use native string operators such as ’like’ with enumerated types (this restriction is not true of every DBMS.)
            \item Using a foreign key reference facilitates modification of the list of acceptable values.
            \item A check constraint is immediately visible in a table definition.
            \item Values shown in the check constraint can be used with native string operators.
            \item A check constraint is not available for use in other tables.
         \end{itemize}

  \section*{NoSQL}
    \subsection*{NoSQL Servers}
      \begin{itemize}
        \item MongoDB: BSON, binary format JSON
        \item MarkLogic: XML with support for JSON and other formats
        \item Apache CouchDB: JSON
        \item Apache Cassandra: key-value store
      \end{itemize}

    \subsection*{Advantages}
      \begin{itemize}
        \item Scaling using clusters of commodity hardware rather than bigger specialist servers
        \item Capacity to handle larger volumes of data and higher transaction rates than rdbms
        \item Less need for database administrators
        \item Lower costs, both to start up and to expand
        \item Few, if any data model restrictions
      \end{itemize}

      \subsection*{Disadvantages}
        \begin{itemize}
          \item NoSQL databases are relatively immature, so expert support can be difficult to obtain
          \item NoSQL data models emphasise whole documents, which eliminates the need for JOINs but which also makes analysis of data sets difficult
          \item There are many NoSQL data manipulation languages, which reduces portability of queries and transferability of skills
          \item The lack of a schema is likely to present problems for maintenance as a database matures
          \item Because data is not normalized, maintaining consistency is challenging
        \end{itemize}
    
      \subsection*{Map reduce}
        \begin{itemize}
          \item A MapReduce job splits input data, in the form of (key,value) pairs, into chunks that are processed in parallel.\

          \item A MapReduce job configuration typically specifies mapping, combination, partitioning, reducing, and input and output formats. A MapReduce job configuration to count the occurrences of words in text files distributed across a network could use the original files as chunks, and specify a map operation to count words, and a reduce operation to combine word counts.\

          \item A map task operates on a single chunk of data, producing as output a collection of (key,value) pairs. A map task in the word counting application could count the words in a single text file, producing a list of words each paired with its number of occurrences.\

          \item A reduce task takes two or more collections of ¡key,value¿ pairs and reduces these to a single collection. A reduce task in the word counting application might take several lists of words with their respective word counts and deliver a single list that gave the total count for each word.
        \end{itemize}

      \subsection*{Eventual Consistency}
        \begin{itemize}
          \item Eventual consistency means that if a data item is not written or updated for a sufficiently long period of time, then all reads of that item will return the same value.
          \item An application developer must allow for the possibility that an application may read data that has been superseded.
          \item Eventual consistency makes it easier to provide readily scalable, highly available distributed systems that continue to operate even when nodes or connections between nodes fail.  
        \end{itemize}
  \section*{Semistructured Data + XML}
    \subsection*{FLOWR}
    FLWOR: for, let, where, order by, return

    for variable_id in document_set 
    let var_id := XPath expression 
    where restriction
    order by ordering
    return result
    \subsection*{Semistructured data}
        Semistructured data is data that has some structure, but whose structure may have some irregularities. Similar entities are grouped together, but entities in the same group need not have the same attributes, and attributes with the same name may have different types.
        Information about the structure of the data may be contained within the data itself (schema-less), but if there is a separate schema describing the structure of the data, that schema only places loose constraints on the data.
        \subsection*{Why use Semistructured data}
        \begin{itemize}
            \item It is useful to be able to treat the Web as a database, but web sources cannot be constrained by a database schema.
            \item A flexible formal facilitates transfer of data between different databases.
            \item XML is very commonly used for data representation and data exchange on the Web, and XML documents are like semistructured data.
        
        \subsection*{XML Document}
        A text that is well-formed according to the XML specification, a fee-free open standard produced by W3C. An XML document is hierarchical, and so is also called an XML tree.
        \subsection*{XML documents like semistructured data}
            \begin{itemize}
                \item both are hierarchical
                \item both can be described by graphs (XML graph is always a tree, though) 
                \item both can be schema-less or self describing
                \item graph for semistructured data need not be a tree, unlike an XML document which always forms a tree
                \item edges are labelled with attribute names in the graph of a semistructured docu- ment, vertices are not labelled; vertices are labelled in an XML tree, edges are not usually labelled
                \item the labels on edges of the graph in the semistructured data model contain schema information; data is stored at leaves; vertices in an XML tree are of different kinds; data is stored at text nodes, other kinds of node provide schema informa- tion, processing instructions and comments.
                \item order of entities and attributes matters in an XML document, not in semistruc- tured data
                \item text and elements can be mixed in XML (see above, re labels)
                \item XML can include comments, processing instructions, entities and more (see above, re labels)
        \end{itemize}

  \section*{Security}
    Areas of concern for data security:\\
    \begin{tabular}{r | p{6cm} | p{6cm} } 
      Confidentiality & Ensure only authorised parties have access to data & Use Encryption, authorisations and authentication\\ & & \\
      Integrity & Ensure data is not modified by authorised or unauthorised parties. and is not corrupt by system limitations & Use checksums, hashes and digital signatures\\ & & \\
      Availability & Ensure the data is accessible when needed & backups, redundancies, attack mitigation\\ & & \\
    \end{tabular}

    \subsection*{Symmetric vs Asymmetric}
      \textbf{Symmetric:}Going one way is the same as going the other. Eg encrypting and decrypting use the same key so must be kept super secure!\\
      \textbf{Asymmetric:}Going one way is not the same as the other. Eg two keys are needed, one to encrypt and one to decrypt. Typically one is ``Private'' and one is ``Public''. Knowing one key does not allow you to go the other way.
    \subsection*{DES - The Data Encryption Standard}
      \begin{itemize}
        \item From IBM Early 1970's, Published and standardized in 80's
        \item 56-bit Keys (with additional 8 bits of parity)
        \item Fast to encrypt/decrypt in hardware and software
        \item superseeded by TDES and AES
        \item Uses sequence of Permutations and Substitutions, repeated 16 times
        \item Produces:
          \begin{itemize}
            \item A Product Cipher - Repeating simple ciphers can produce more complex one
            \item A Bloc Cipher - Operates on a block of data (can be adapted to streams)
            \item A Symmetric-key Cipher - Encrypting and Decrypting keys are the same
          \end{itemize}
      \end{itemize}

    \subsection*{}


\end{document}

