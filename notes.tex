\documentclass[10pt]{article}
\usepackage{a4wide}
\usepackage[english]{babel}
\usepackage{fancyhdr}
\usepackage{hyperref}
\usepackage{lastpage}
\usepackage{graphicx}
\usepackage[section]{placeins}
\usepackage[superscript,biblabel]{cite}
\usepackage[margin=1in]{geometry}

\pagestyle{fancy}
\rhead{}

\begin{document}
  \section*{Relational Data Model}
    \subsection*{Definitions}
    \begin{itemize}
      \item An entity class is a collection of entity instances that have a common structure.
      For example, a whole collection of student records could form an entity class.

      \item An entity instance represents a particular object of interest that is to be represented
      and tracked. For example, a student record is an entity instance that
      represents an individual student.

      \item An attribute represents a piece of interesting information, or a measurable fact,
      about the instances of an entitity class. For example, ’year of first registration’
      is a fact about students that might be represented as an attribute of all the
      instances of the entity class ’student’

      \item A domain is a set of values that can be assigned to an attribute; for example,
      the attribute ’birthday’ coudl be given values from the domain ’date’.

      \item A relationship is an association between entities. Entities are often identified by
      nouns in a requirements specification, and relationships by verbs. For example,
      ’owns’ might form a relationship between entities ’person’ and ’vehicle’. Relationships
      can be described as relationships between entity classes, or between
      entity instances.

      \item Mathematically, a relation consists of a heading, which is a subset of the Cartesian
      product of a set of (attribute name, domain) pairs, and a body, which contains
      (attribute name, value) pairs. For example, the entity class student could be
      represented as a heading, (student number, integer), (student name, text) and
      a body containing values like (student number, 123),(student name, Bloggs)
      A relation is implemented as a table in a relational database.

      \item A candidate key is a minimal set of attributes that identifies each individual
      row in a table (each tuple in a relation). For example, suppose there was a
      relation Slotroom,day,time in a timetabling application. Then room,day,time or
      class,day,time would serve as alternative candidate keys for the relation.

      \item The primary key is the candidate key that has been nominated to identify individual
      rows in a table. For example, in the timetabling relation above, room,day,time
      would be likely to form a suitable primary key because ’class’ is likely to change.
    \end{itemize}


  \section*{Database Integrity}
    \subsection*{ACID}
      \begin{itemize}
        \item Atomicity - something is either done completely, or not done at all. The state of doing it is not visible outside the database.
        \item Consistency - The database is in a legal state at all times. When a transaction occurs, it can not break the rules. These rules are about integrity, what is allowed and what is not allowed in certain locations of the database.
        \item Isolation - There can be more than one transaction occurring at the same time. A certain transaction will not see changes made by other transactions.
        \item Durability - When a transaction is done, it will be committed. After it is committed, it can no longer be undone.
      \end{itemize}

  \section*{NoSQL}

  \section*{Semistructured Data + XML}
    \subsection*{FLOWR}
       FOR, LET, WHERE, ORDER BY, RETURN.

  \section*{Security}
    Areas of concern for data security:\\
    \begin{tabular}{r | p{6cm} | p{6cm} } 
      Confidentiality & Ensure only authorised parties have access to data & Use Encryption, authorisations and authentication\\ & & \\
      Integrity & Ensure data is not modified by authorised or unauthorised parties. and is not corrupt by system limitations & Use checksums, hashes and digital signatures\\ & & \\
      Availability & Ensure the data is accessible when needed & backups, redundancies, attack mitigation\\ & & \\
    \end{tabular}

    \subsection*{Symmetric vs Asymmetric}
      \textbf{Symmetric:}Going one way is the same as going the other. Eg encrypting and decrypting use the same key so must be kept super secure!\\
      \textbf{Asymmetric:}Going one way is not the same as the other. Eg two keys are needed, one to encrypt and one to decrypt. Typically one is ``Private'' and one is ``Public''. Knowing one key does not allow you to go the other way.
    \subsection*{DES - The Data Encryption Standard}
      \begin{itemize}
        \item From IBM Early 1970's, Published and standardized in 80's
        \item 56-bit Keys (with additional 8 bits of parity)
        \item Fast to encrypt/decrypt in hardware and software
        \item superseeded by TDES and AES
        \item Uses sequence of Permutations and Substitutions, repeated 16 times
        \item Produces:
          \begin{itemize}
            \item A Product Cipher - Repeating simple ciphers can produce more complex one
            \item A Bloc Cipher - Operates on a block of data (can be adapted to streams)
            \item A Symmetric-key Cipher - Encrypting and Decrypting keys are the same
          \end{itemize}
      \end{itemize}

    \subsection*{}


\end{document}

